% !TeX encoding = UTF-8
% !TeX program = XeLaTeX
% !TeX spellcheck = en_US

% Author : Jason Jia and Narsil Zhang
% Description : Outline: PyTorch --- Seminar on Selected Tools Week 4

\documentclass[english, nochinese]{../TeXTemplate/pkuslide}

\usepackage{../TeXTemplate/def}

\title{Outline: PyTorch}
\subtitle{Seminar on Selected Tools Week 4 --- PyTorch}

\author{Jason Jia\qquad Narsil Zhang}
\date{Updated on March 26}

\subject{Outline: PyTorch --- Seminar on Selected Tools Week 4}
\keywords{}

\begin{document}

\begin{frame}
\titlepage
\end{frame}

\begin{frame}
\tableofcontents[subsectionstyle=show]
\end{frame}

\section{Introduction}

\begin{frame}
\sectionpage
\end{frame}

\begin{frame}{Introduction}
PyTorch is a python package that provides two high-level features:
\begin{itemize}
\item Tensor computation (like numpy) with strong GPU acceleration
\item Deep Neural Networks built on a tape-based autograd system
\end{itemize}
For more information, you can visit \url{http://pytorch.org/}
\end{frame}

\section{Tensor}

\begin{frame}
\sectionpage
\end{frame}

\begin{frame}{Construction}
\begin{enumerate}
    \item torch.Tensor(2,3)
    \item torch.ones(2,3)
    \item torch.zeros(2,3)
    \item torch.eye(2,3)
    \item torch.rand(2,3)
    \item torch.range(6)
\end{enumerate}
\end{frame}

\begin{frame}{Common Operation}
\begin{enumerate}
    \item view
	\item unsqueeze
	\item squeeze
	\item expand
	\item sum/mean/median/mode
\end{enumerate}
\end{frame}

\begin{frame}{Type}
\begin{enumerate}
    \item FloatTensor
	\item DoubleTensor
	\item ByteTensor
	\item CharTensor
	\item ShortTensor
	\item IntTensor
	\item LongTensor
\end{enumerate}
\end{frame}

\begin{frame}{Transfer from and to NumPy}
\begin{enumerate}
	\item t = torch.from\_numpy(n): Transfer a PyTorch Tensor from a NumPy array
	\item t = torch.Tensor(n): Transfer a NumPy array into a PyTorch Tensor
	\item n = t.numpy(): Transfer a PyTorch Tensor from a NumPy array
\end{enumerate}
\end{frame}

\section{Autograd}

\begin{frame}
\sectionpage
\end{frame}


\begin{frame}{Variable}
\begin{itemize}
	\item Method to import: "from torch.autograd import Variable"
	\item Obtain Data from Variables: (Variable).data
	\item requires\_grad = True/False
	\item Back Propagation: (Variable).backward()
	\item Calculating the Graph
\end{itemize}
\end{frame}

\section{Neural Networks: torch.nn}

\begin{frame}
\sectionpage
\end{frame}

\begin{frame}{Layers}
\begin{itemize}
	\item Convolutional Layers: nn.Conv1d/Conv2d/Conv3d
	\item Pooling Layers: nn.MaxPool2d
	\item Linear Layers: nn.Linear
	\item ReLU (type of non-linear function) Layers: nn.ReLU
	\item Batch Normalization Layers: nn.BatchNorm1d
\end{itemize}
\end{frame}

\begin{frame}{Loss Functions}
\begin{itemize}
	\item Cross Entropy Loss: nn.CrossEntropyLoss
	\item Mean Squared Error: nn.MSELoss
	\item KL Divergence: nn.KLDivLoss
\end{itemize}
\end{frame}

\section{Optimizer: torch.optim}

\begin{frame}
\sectionpage
\end{frame}

\begin{frame}{Optimizer}
\begin{itemize}
	\item Stochastic Gradient Descent (Momentum): optim.SGD
	\item RMSProp: optim.Rprop
	\item Adadelta: optim.Adadelta
	\item Adagrad: optim.Adagrad
	\item Adam: optm.Adam
\end{itemize}
\end{frame}
	
\section{Date Import and Management}

\begin{frame}
\sectionpage
\end{frame}

\begin{frame}{Constructing Datasets}
\begin{itemize}
	\item torch.utils.DateLoader
	\item torchversion.datasets
	\item torchversion.transforms
\end{itemize}
\end{frame}

\end{document}
