% !TeX encoding = UTF-8
% !TeX program = XeLaTeX
% !TeX spellcheck = en_US

% Author : Jason Jia & Narsil Zhang
% Description : Materials: PyTorch --- Seminar on Selected Tools Week 4

\documentclass[english]{../TeXTemplate/pkupaper}

\usepackage[paper]{../TeXTemplate/def}

\newcommand{\cuniversity}{}
\newcommand{\cthesisname}{Materials: PyTorch --- Seminar on Selected Tools Week 4}
\newcommand{\titlemark}{Materials: PyTorch}

\title{\titlemark}
\author{Jason Jia\and Narsil Zhang}
\date{Updated on March 27}

\begin{document}

\maketitle
\section{Installation and Configuration}

With the support of GPU-acceleration tensor computation and its autograd system, PyTorch has now become one of the most popular deeplearning frameworks. PyTorch consists of two main Python package--\verb"torch" and \verb"torchvision" (the latter consists of popular datasets, model architectures, and common image transformations for computer vision). Let's see how to install them.

For Mac or Linux users, it's suggested to follow the instruction on the \href{http://pytorch.org}{official website of PyTorch}. You can choose the version of Python it correspond to and whether to install a GPU-available version of PyTorch at you own will. \emph{An out-of-the-box command will be given, corresponding to your environment.}

For Windows users things may be a little different. The method in \href{https://zhuanlan.zhihu.com/p/26871672}{this website} has been proved useful. Notice that you should install torchvision using pip after the installation of PyTorch, that is, \verb"pip install torchvision".

\section{Resources}

A series of tutorials on \href{http://pytorch.org}{\emph{pytorch.org}} are recommended, especially the \href{http://pytorch.org/tutorials/beginner/deep_learning_60min_blitz.html}{\emph{"Deep Learning with PyTorch: A 60 Minute Blitz"}}. A Chinese version of it can be found \href{https://zhuanlan.zhihu.com/p/25572330}{here}.

\href{https://zhuanlan.zhihu.com/c_94953554}{This special column} in Zhihu can also serve as a clear tutorial.

Once you get in some trouble concerning PyTorch, you can search \href{http://pytorch.org/docs/master/index.html}{the Docs of PyTorch} for details. Alternatively you can try to find if there are similar problems on \href{https://discuss.pytorch.org/}{\emph{PyTorch Discuss}}. 

\section{Assignment}

\begin{partlist}
\item (Required) Write a linear regression model using PyTorch. (See folder "Linear\_Regression")
\item (Required) Write a one-hidden-layer neural network to classify the MNIST data. (See folder "MNIST\_Classifier")
\item (Optional) (Requires deep learning knowledge) Implement convolutional neural network and try to train this CNN to classify Cifar-10 data (Source: PyTorch tutorial). (See folder "CNN")
\item (Optional) (Requires deep learning knowledge) Complete the neural style transfer code and try to generate some beautiful and artistic pictures by yourself (Source: Stanford Course CS231n). (See folder "Neural\_Style\_Transfer)
\item (Optional) (Requires deep learning knowledge) Write your own neural network. Suggestions include ResNet, LSTM, GAN, and some RL methods such as DQN.
\end{partlist}

\par For all the assignment above (except for the last one), you can either complete the code provided in "program" folder, or try to implement on your own.

\end{document}
